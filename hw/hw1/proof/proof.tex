\documentclass[11pt]{ctexart}
\usepackage{amsmath,amssymb,amsthm,bm}
\usepackage{fullpage}
\usepackage{hyperref}

\title{CS189 作业:硬间隔支持向量机(理论题)中文题面与证明}
\author{}
\date{}

\newtheorem{problem}{问题}
\newtheorem*{solution}{证明}

\begin{document}
\maketitle

\section*{背景与记号}
线性 SVM 的判别函数(原文式(1))为
\[
r(x)=\begin{cases}
+1,& w^\top x+\alpha\ge 0,\\
-1,& \text{otherwise}.
\end{cases}
\]
原始(primal)硬间隔 SVM(原文式(2)):
\[
\min_{w,\alpha}\ \|w\|^2\quad
\text{s.t.}\quad y_i\big(X_i^\top w+\alpha\big)\ge 1,\ i=1,\dots,n.
\]
其拉格朗日鞍点形式(原文式(3)):
\[
\max_{\lambda_i\ge 0}\ \min_{w,\alpha}\ \|w\|^2-\sum_{i=1}^n
\lambda_i\big(y_i(X_i^\top w+\alpha)-1\big).
\]

\begin{problem}
(a)把式(3)化为对偶优化(原文式(4)),并解释新等式约束的来源。
\end{problem}

\begin{solution}
设
\[
\mathcal L(w,\alpha,\lambda)=\|w\|^2-\sum_{i=1}^n
\lambda_i\big(y_i(X_i^\top w+\alpha)-1\big),\qquad \lambda_i\ge 0.
\]
对内层最小化的一阶条件(站立性)为
\begin{align*}
\frac{\partial\mathcal L}{\partial w}&=2w-\sum_{i=1}^n\lambda_i y_i X_i=0
\ \Rightarrow\
w^{*}=\frac12\sum_{i=1}^n\lambda_i y_i X_i,\\
\frac{\partial\mathcal L}{\partial \alpha}&=-\sum_{i=1}^n\lambda_i y_i=0
\ \Rightarrow\
\sum_{i=1}^n\lambda_i y_i=0.
\end{align*}
将 $w^{*}$ 与等式约束代回 $\mathcal L$,注意
\[
\|w^{*}\|^2
=\Big(\tfrac12\sum_i\lambda_i y_i X_i\Big)^\top
 \Big(\tfrac12\sum_j\lambda_j y_j X_j\Big)
=\frac14\sum_{i,j}\lambda_i\lambda_jy_iy_j\,X_i^\top X_j,
\]
并且 $-\alpha\sum_i\lambda_i y_i=0$,故对偶函数为
\[
g(\lambda)=\sum_{i=1}^n\lambda_i-\frac14\sum_{i=1}^n\sum_{j=1}^n
\lambda_i\lambda_jy_iy_j\,X_i^\top X_j.
\]
弱对偶性保证 $g(\lambda)$ 是原始问题最优值的下界,故对偶问题为
\[
\max_{\lambda\ge 0}\ g(\lambda)
\quad \text{s.t.}\quad \sum_{i=1}^n\lambda_i y_i=0.
\]
新等式约束正是由 $\partial\mathcal L/\partial\alpha=0$ 得到。证毕。
\end{solution}

\begin{problem}
(b)已知能最优解式(3)的 $\lambda_i^{*}$ 与 $\alpha^{*}$。证明由式(1)给出的判别函数可写为
\[
r(x)=\operatorname{sign}\!\left(\alpha^{*}+\frac12\sum_{i=1}^n \lambda_i^{*} y_i\,X_i^\top x\right).
\]
\end{problem}

\begin{solution}
由(a)的平稳性条件,$w^{*}=\tfrac12\sum_i\lambda_i^{*} y_i X_i$。代回
\[
r(x)=\operatorname{sign}(w^{*\top}x+\alpha^{*})
=\operatorname{sign}\!\left(\alpha^{*}+\tfrac12\sum_{i=1}^n \lambda_i^{*} y_i\,X_i^\top x\right).
\]
证毕。
\end{solution}

\begin{problem}
(c)利用互补松弛条件 $\lambda_i^{*}\big(y_i(X_i^\top w^{*}+\alpha^{*})-1\big)=0$,
解释当 $\lambda_i^{*}>0$ 时样本点 $X_i$ 的几何意义。
\end{problem}

\begin{solution}
若 $\lambda_i^{*}>0$,则必有 $y_i(X_i^\top w^{*}+\alpha^{*})=1$,即该点恰落在两条
间隔超平面之一 $\{x:\ w^{*\top}x+\alpha^{*}=\pm 1\}$ 上,因而是支持向量。证毕。
\end{solution}

\begin{problem}
(d)说明在计算 $r(x)$ 时,只需使用 $\lambda_i^{*}>0$ 的样本点。
\end{problem}

\begin{solution}
由(b)
\[
w^{*\top}x+\alpha^{*}=\alpha^{*}+\frac12\sum_{i=1}^n\lambda_i^{*} y_i\,X_i^\top x.
\]
若某 $i$ 使 $\lambda_i^{*}=0$,其对和式无贡献;只有 $\lambda_i^{*}>0$(支持向量)
才影响判别值,故仅需支持向量即可计算 $r(x)$。证毕。
\end{solution}

\begin{problem}
(e)给定近似参数 $w=[-0.4528,\,-0.5190]^\top,\ \alpha=0.1471$。
写出二维平面上决策边界 $w^\top x+\alpha=0$ 及两条边距 $w^\top x+\alpha=\pm 1$ 的解析式,
并说明支持向量所在位置。
\end{problem}

\begin{solution}
令 $x=(x_1,x_2)^\top$。决策边界为
\[
x_2=-\frac{w_1}{w_2}x_1-\frac{\alpha}{w_2}\approx -0.8723\,x_1-0.2834,
\]
两条边距为
\[
x_2=-\frac{w_1}{w_2}x_1-\frac{\alpha\pm 1}{w_2}.
\]
支持向量满足 $w^\top X_i+\alpha=\pm 1$,即恰位于上述两条平行直线上。证毕。
\end{solution}

\begin{problem}
(f)在线性可分情形下,用原始形式证明:最优解的每个类别($+1$ 与 $-1$)至少各有一个支持向量。
\end{problem}

\begin{solution}
反证法。设最优解 $(w^{*},\alpha^{*})$ 下,正类无支持向量。则存在 $\varepsilon>0$,
对所有 $y_i=+1$ 有 $w^{*\top}X_i+\alpha^{*}\ge 1+\varepsilon$。取
\[
\tilde\alpha=\alpha^{*}-\varepsilon/2.
\]
则正类满足 $w^{*\top}X_i+\tilde\alpha\ge 1+\varepsilon/2>1$。
对负类 $y_i=-1$,有
\[
y_i(w^{*\top}X_i+\tilde\alpha)=-(w^{*\top}X_i+\alpha^{*})+\varepsilon/2\ge 1+\varepsilon/2>1.
\]
于是存在 $\delta=\varepsilon/2>0$ 使所有样本均满足
$y_i(w^{*\top}X_i+\tilde\alpha)\ge 1+\delta$。对 $(w^{*},\tilde\alpha)$ 同比缩放
\[
w'=\frac{w^{*}}{1+\delta/2},\qquad \alpha'=\frac{\tilde\alpha}{1+\delta/2},
\]
则 $y_i(w'^\top X_i+\alpha')\ge \frac{1+\delta}{1+\delta/2}>1$,故 $(w',\alpha')$ 仍可行,
但目标 $\|w'\|^2=\|w^{*}\|^2/(1+\delta/2)^2<\|w^{*}\|^2$,与最优性矛盾。
故正类必须存在支持向量;对负类同理,结论成立。证毕。
\end{solution}

\end{document}
