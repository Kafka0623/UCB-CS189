%!TeX program = xelatex
\documentclass[10.5pt,hyperref,a4paper,UTF8]{ctexart}
\usepackage{RUCReport}
\usepackage{listings}
\usepackage{xcolor}
\usepackage{threeparttable}
\usepackage{booktabs}
\usepackage{array}
\usepackage{subcaption}
\usepackage{graphicx}
\usepackage{geometry}
\usepackage{float}
\geometry{a4paper, margin=1in}
\renewcommand\thesubfigure{\arabic{subfigure}} % 子图编号改为中文小写
% 定义可能使用到的颜色
\definecolor{CPPLight}  {HTML} {686868}
\definecolor{CPPSteel}  {HTML} {888888}
\definecolor{CPPDark}   {HTML} {262626}
\definecolor{CPPBlue}   {HTML} {4172A3}
\definecolor{CPPGreen}  {HTML} {487818}
\definecolor{CPPBrown}  {HTML} {A07040}
\definecolor{CPPRed}    {HTML} {AD4D3A}
\definecolor{CPPViolet} {HTML} {7040A0}
\definecolor{CPPGray}  {HTML} {B8B8B8}
\definecolor{keywordcolor}{rgb}{0.8,0.1,0.5}
\definecolor{webgreen}{rgb}{0,.5,0}
\definecolor{bgcolor}{rgb}{0.92,0.92,0.92}

\makeatletter
\renewcommand\@cite[2]{\textsuperscript{[#1]}} % 将引用变为右上角的形式
\makeatother

\hypersetup{
    citecolor=black,  % 修改文献引用的颜色为黑色
}

\lstset{
    breaklines = true,                                   % 自动将长的代码行换行排版
    extendedchars=false,                                 % 解决代码跨页时,章节标题,页眉等汉字不显示的问题
    columns=fixed,       
    numbers=left,                                        % 在左侧显示行号
    basicstyle=\zihao{-5}\ttfamily,
    numberstyle=\small,
    frame=none,                                          % 不显示背景边框
    % backgroundcolor=\color[RGB]{245,245,244},            % 设定背景颜色
    keywordstyle=\color[RGB]{40,40,255},                 % 设定关键字颜色
    numberstyle=\footnotesize\color{darkgray},           % 设定行号格式
    commentstyle=\it\color[RGB]{0,96,96},                % 设置代码注释的格式
    stringstyle=\rmfamily\slshape\color[RGB]{128,0,0},   % 设置字符串格式
    showstringspaces=false,                              % 不显示字符串中的空格
    % frame=leftline,topline,rightline, bottomline         %分别对应只在左侧,上方,右侧,下方有竖线
    frame=shadowbox,                                     % 设置阴影
    rulesepcolor=\color{red!20!green!20!blue!20},        % 阴影颜色
    basewidth=0.6em,
}

\lstdefinestyle{CPP}{
    language=c++,                                        % 设置语言
    morekeywords={alignas,continute,friend,register,true,alignof,decltype,goto,
    reinterpret_cast,try,asm,defult,if,return,typedef,auto,delete,inline,short,
    typeid,bool,do,int,signed,typename,break,double,long,sizeof,union,case,
    dynamic_cast,mutable,static,unsigned,catch,else,namespace,static_assert,using,
    char,enum,new,static_cast,virtual,char16_t,char32_t,explict,noexcept,struct,
    void,export,nullptr,switch,volatile,class,extern,operator,template,wchar_t,
    const,false,private,this,while,constexpr,float,protected,thread_local,
    const_cast,for,public,throw,std},
    emph={map,set,multimap,multiset,unordered_map,unordered_set,
    unordered_multiset,unordered_multimap,vector,string,list,deque,
    array,stack,forwared_list,iostream,memory,shared_ptr,unique_ptr,
    random,bitset,ostream,istream,cout,cin,endl,move,default_random_engine,
    uniform_int_distribution,iterator,algorithm,functional,bing,numeric,},
    emphstyle=\color{CPPViolet}, 
}

\lstdefinestyle{Java}{
    language=[AspectJ]Java,
    keywordstyle=\color{keywordcolor}\bfseries
}

\lstdefinestyle{Python}{
    language=Python,
}
\newtheorem{theorem}{定理}
\newtheorem{lemma}{引理}
\newtheorem{proposition}{命题}
\newtheorem{definition}{定义}
\newcommand{\R}{\mathbb{R}}
\newcommand{\sign}{\operatorname{sign}}

%%-------------------------------正文开始---------------------------%%
\begin{document}

%%-----------------------封面--------------------%%
\cover

%%------------------摘要-------------%%
\begin{abstract}
%
XXXXX\\
\textbf{关键词:}XXX
%
\end{abstract}

%%--------------------------目录页------------------------%%
\newpage
\tableofcontents
\thispagestyle{empty} % 目录不显示页码

%%------------------------正文页从这里开始-------------------%
\newpage
\setcounter{page}{1} % 让页码从正文开始编号

%%可选择这里也放一个标题
\begin{center}
    \title{ \Huge \textbf{} }
\end{center}


\thispagestyle{empty} % 首页不显示页码


\section{基本设定与增广记号}
\begin{enumerate}
  \item 训练集 $\{(x_i,y_i)\}_{i=1}^n$, $x_i\in\R^d$, $y_i\in\{\pm1\}$;线性分类器 $f(x)=w\cdot x+b$,超平面 $H=\{x:w\cdot x+b=0\}$。
  \item 增广向量:$\tilde x=(x,1)\in\R^{d+1}$,$\tilde w=(w,b)\in\R^{d+1}$,则 $f(x)=\tilde w\cdot\tilde x$;在 $d{+}1$ 维里用过原点超平面统一表示一般超平面。
  \item 预测 $\hat y(x)=\sign(f(x))$;分类约束 $y_i(w\cdot x_i+b)\ge 0$。
\end{enumerate}

\section{$x$-space 与 $w$-space 的几何对偶}
\subsection{点–超平面对偶与分类约束的两种视角}
\begin{proposition}[对偶关系]
在 $x$-space:超平面 $\{z:w\cdot z+b=0\}$ 对应法向量 $w$;在 $w$-space:样本 $x$ 诱导超平面 $\{z:x\cdot z+b=0\}$。存在等价式
\[
x\in\{z:w\cdot z+b=0\}\iff w\in\{z:x\cdot z+b=0\}.
\]
\end{proposition}
\noindent 结论:把“在 $x$-space 找超平面”转化为“在 $w$-space 找点”。分类约束 $y_i(w\cdot x_i+b)\ge 0$ 在 $x$-space 表示样本在边界正确侧;在 $w$-space 表示 $w$ 在样本诱导超平面的正确侧。所有半空间的交集为 $w$ 的可行域。

\subsection{“$w$ 既是向量又是点”的说明}
参数 $w=(w_1,\dots,w_d)$ 在 $w$-space 中是点的坐标;在 $x$-space 中表现为法向量。优化问题“找 $w$”即“在 $\R^d$ 中找点”。

\section{风险函数、形状与梯度}
\subsection{仅误分类计损的风险与感知机损失}
\begin{enumerate}
  \item 仅误分类计损的风险函数
  \[
  R(w,b)=\sum_{i\in V}-y_i\,(w\cdot x_i+b),\quad V=\{i:\ y_i(w\cdot x_i+b)<0\}.
  \]
  它在 $w$-space 呈分段线性凸“折面”,折痕与样本诱导直线对齐;$R=0$ 的底部扇区即全体正确分类解集。
  \item 感知机(凸)损失 $L_i(w,b)=\max(0,-y_i(w\cdot x_i+b))$,经验风险 $F(w,b)=\frac1n\sum_i L_i$;两者在误分类处的(次)梯度方向一致。
\end{enumerate}

\subsection{梯度、最速下降与更新}
\begin{enumerate}
  \item 多元一阶泰勒 $f(w+\Delta)\approx f(w)+\nabla f(w)^\top\Delta$;令 $\Delta=-\eta\nabla f(w)$ 得
  \[
  w^{(t+1)}=w^{(t)}-\eta\,\nabla f\big(w^{(t)}\big).
  \]
  \item 对 $R$ 的梯度:$\nabla_w R(w)=-\sum_{i\in V}y_i x_i,\ \partial_b R(w)=-\sum_{i\in V}y_i$。对 $L_i$:若 $y_i(w\cdot x_i+b)<0$,则 $\partial_w L_i=-y_i x_i,\ \partial_b L_i=-y_i$,否则为 $0$。
\end{enumerate}

\section{GD 与 SGD:定义、复杂度与在线性}
\subsection{两种下降}
\begin{enumerate}
  \item 全量 GD:$w\leftarrow w-\eta\,\nabla F(w)$,每步 $O(nd)$,方向平稳。
  \item SGD:随机抽样 $i$,$w\leftarrow w-\eta\,\nabla L_i(w)$,每步 $O(d)$;由于 $F=\frac1n\sum_i L_i$,有 $\mathbb E[\nabla L_i(w)]=\nabla F(w)$,单步有噪声但无偏。
\end{enumerate}
\subsection{在线算法与步长}
\begin{enumerate}
  \item 感知机是在线算法:新样本到来可继续迭代。
  \item 步长 $\eta$ 不出现在错误次数的大 $O$ 上界,但实际运行时间对 $\eta$ 敏感:$\eta$ 太小则慢;太大则可能跨越零风险区、在边界两侧振荡。可用线搜索、自适应步长或直接改用二次规划。
\end{enumerate}

\section{偏置与“增加一维”的统一}
\begin{enumerate}
  \item 一般超平面 $w\cdot x+\alpha=0$ 可通过增广 $\tilde x=(x,1)$、$\tilde w=(w,\alpha)$ 统一为 $\tilde w\cdot\tilde x=0$,从而用一条更新式覆盖 $w$ 与偏置。
  \item 在 $d{+}1$ 维空间中,训练点共面于 $x_{d+1}=1$ 的切片,仍可直接运行感知机或其他线性分类算法。
\end{enumerate}

\section{凸优化与 SGD 收敛}
\begin{enumerate}
  \item 最小化凸函数:局部极小即全局极小;合适步长(如衰减序列)下,SGD 在凸光滑情形以期望收敛。
  \item 非凸目标(如深度网络)通常无法保证全局最优,但 SGD 仍实用;在本课的线性可分与感知机损失场景下,存在更强的有限步收敛结论。
\end{enumerate}

\section{感知机收敛定理与错误上界}
\subsection{陈述与参数}
定义半径 $R=\max_i\|\tilde x_i\|$。若存在单位向量 $\|\tilde w^\star\|=1$ 使得几何间隔下界
\[
\gamma=\min_i y_i(\tilde w^\star\cdot\tilde x_i)>0,
\]
则在线更新(仅误分类样本)$\tilde w\leftarrow\tilde w+\eta\,y_i\tilde x_i$ 的犯错次数 $T$ 满足
\[
T\le\Big(\frac{R}{\gamma}\Big)^2\frac{1}{\eta^2}.
\]
常用归一化 $R\le1,\ \eta=1$ 时,$T\le 1/\gamma^2$。

\subsection{证明(Novikoff)}
\begin{enumerate}
  \item 投影线性累增:每次犯错使 $\langle\tilde w^{(t+1)},\tilde w^\star\rangle\ge \langle\tilde w^{(t)},\tilde w^\star\rangle+\eta\gamma$,累加得 $\langle\tilde w^{(T)},\tilde w^\star\rangle\ge T\eta\gamma$。
  \item 范数二次受限:犯错时 $y_t(\tilde w^{(t)}\cdot\tilde x_t)<0$,故 $\|\tilde w^{(t+1)}\|^2\le \|\tilde w^{(t)}\|^2+\eta^2\|\tilde x_t\|^2\le \|\tilde w^{(t)}\|^2+\eta^2R^2$,从而 $\|\tilde w^{(T)}\|\le \eta R\sqrt{T}$。
  \item Cauchy--Schwarz:$T\eta\gamma\le \|\tilde w^{(T)}\|\le \eta R\sqrt{T}$,解得结论。
\end{enumerate}

\section{间隔(margin)的本质与最大化间隔}
\subsection{函数间隔与几何间隔}
\begin{definition}[函数间隔]
$\hat\gamma_i=y_i(w\cdot x_i+\alpha)$,受 $w$ 的缩放影响。
\end{definition}
\begin{definition}[几何间隔]
$\gamma_i=\dfrac{y_i(w\cdot x_i+\alpha)}{\|w\|}$ 为点到超平面的带符号欧氏距离,整体间隔 $\gamma=\min_i\gamma_i$。
\end{definition}

\subsection{定标、条带与最大化}
为消除缩放不定性并排除 $w=0$,施加
\[
y_i(w\cdot x_i+\alpha)\ge 1,\quad i=1,\dots,n.
\]
此时最近样本(支持向量)满足等式,故
\[
\gamma=\frac{1}{\|w\|},\qquad \text{条带宽度}=2/\|w\|,\quad \text{中线 }w\cdot x+\alpha=0.
\]
最大化间隔等价于最小化 $\|w\|$(实际用光滑的 $\tfrac12\|w\|^2$)。

\section{硬间隔 SVM 的凸二次规划与权重空间几何}
\subsection{原始问题(线性可分)}
\[
\boxed{\ \min_{w,\alpha}\ \frac12\|w\|^2\quad \text{s.t.}\quad y_i(w\cdot x_i+\alpha)\ge 1,\ i=1,\dots,n.\ }
\]
\begin{enumerate}
  \item 目标严格凸、约束线性,线性可分时解唯一;最优几何间隔为 $1/\|w\|$。
  \item 采用 $\|w\|^2$ 而非 $\|w\|$ 的理由是可微性与计算便利,解不变。
\end{enumerate}

\subsection{权重空间三维/截面直观}
\begin{enumerate}
  \item 在 $(w_1,w_2,\alpha)$ 空间,每个训练点诱导一个线性约束平面;可行域为这些半空间交集。
  \item 目标只惩罚 $w$ 的长度(到 $\alpha$ 轴的水平距离),最优解是可行域中离 $\alpha$ 轴最近的点;二维截面中呈若干直线交点(支持向量)给定的解。
\end{enumerate}

\section{从感知机到 SVM:动机与实践}
\begin{enumerate}
  \item 感知机只求可分,解不唯一、对步长敏感、整体仍可能慢;SVM 通过最大化间隔给出唯一且鲁棒的解,二次规划求解更稳定高效。
  \item 数据不可分时,引入软间隔 $y_i(w\cdot x_i+\alpha)\ge 1-\xi_i,\ \xi_i\ge0$ 并最小化 $\frac12\|w\|^2+C\sum_i\xi_i$;进一步可用核方法得到非线性边界。
\end{enumerate}

\section{历史与备注}
\begin{enumerate}
  \item Rosenblatt(1957)在康奈尔航空实验室提出并以硬件实现 Mark I 感知机(用于 $20\times20$ 像素图像),感知机是深度学习历史的起点之一。
  \item 现代优化与统计学习理论推动了从“可分即可”到“最大间隔与泛化”的转变。
\end{enumerate}

\section{常见疑问与澄清}
\begin{enumerate}
  \item “最优点是否一定在超平面交点”:最优解位于可行域边界;在 $d$ 维可为若干约束的交点/交线/交面。
  \item “是否降维”:属于解集自由度因约束而降低,并非数据维度降维(不同于 PCA)。
  \item “为何用 $-\nabla f$ 更新”:来自多元一阶泰勒与最速下降方向。
  \item “单样本梯度为何无偏”:$F=\frac1n\sum_i L_i \Rightarrow \nabla F=\frac1n\sum_i\nabla L_i$,均匀抽样 $\mathbb E[\nabla L_i]=\nabla F$。
  \item “为何右端取 $1$ 而非 $0$”:消除缩放不定性、排除 $w=0$,并让几何间隔与 $\|w\|$ 建立一一关系,从而把“最大化间隔”转为“最小化 $\|w\|$”。
\end{enumerate}



%%----------- 参考文献 -------------------%%
%在reference.bib文件中填写参考文献,此处自动生成
\newpage
\reference


\end{document}