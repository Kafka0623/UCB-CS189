%!TeX program = xelatex
\documentclass[10.5pt,hyperref,a4paper,UTF8]{ctexart}
\usepackage{RUCReport}
\usepackage{listings}
\usepackage{xcolor}
\usepackage{threeparttable}
\usepackage{booktabs}
\usepackage{array}
\usepackage{subcaption}
\usepackage{graphicx}
\usepackage{geometry}
\usepackage{float}
\geometry{a4paper, margin=1in}
\renewcommand\thesubfigure{\arabic{subfigure}} % 子图编号改为中文小写
% 定义可能使用到的颜色
\definecolor{CPPLight}  {HTML} {686868}
\definecolor{CPPSteel}  {HTML} {888888}
\definecolor{CPPDark}   {HTML} {262626}
\definecolor{CPPBlue}   {HTML} {4172A3}
\definecolor{CPPGreen}  {HTML} {487818}
\definecolor{CPPBrown}  {HTML} {A07040}
\definecolor{CPPRed}    {HTML} {AD4D3A}
\definecolor{CPPViolet} {HTML} {7040A0}
\definecolor{CPPGray}  {HTML} {B8B8B8}
\definecolor{keywordcolor}{rgb}{0.8,0.1,0.5}
\definecolor{webgreen}{rgb}{0,.5,0}
\definecolor{bgcolor}{rgb}{0.92,0.92,0.92}

\makeatletter
\renewcommand\@cite[2]{\textsuperscript{[#1]}} % 将引用变为右上角的形式
\makeatother

\hypersetup{
    citecolor=black,  % 修改文献引用的颜色为黑色
}

\lstset{
    breaklines = true,                                   % 自动将长的代码行换行排版
    extendedchars=false,                                 % 解决代码跨页时,章节标题,页眉等汉字不显示的问题
    columns=fixed,       
    numbers=left,                                        % 在左侧显示行号
    basicstyle=\zihao{-5}\ttfamily,
    numberstyle=\small,
    frame=none,                                          % 不显示背景边框
    % backgroundcolor=\color[RGB]{245,245,244},            % 设定背景颜色
    keywordstyle=\color[RGB]{40,40,255},                 % 设定关键字颜色
    numberstyle=\footnotesize\color{darkgray},           % 设定行号格式
    commentstyle=\it\color[RGB]{0,96,96},                % 设置代码注释的格式
    stringstyle=\rmfamily\slshape\color[RGB]{128,0,0},   % 设置字符串格式
    showstringspaces=false,                              % 不显示字符串中的空格
    % frame=leftline,topline,rightline, bottomline         %分别对应只在左侧,上方,右侧,下方有竖线
    frame=shadowbox,                                     % 设置阴影
    rulesepcolor=\color{red!20!green!20!blue!20},        % 阴影颜色
    basewidth=0.6em,
}

\lstdefinestyle{CPP}{
    language=c++,                                        % 设置语言
    morekeywords={alignas,continute,friend,register,true,alignof,decltype,goto,
    reinterpret_cast,try,asm,defult,if,return,typedef,auto,delete,inline,short,
    typeid,bool,do,int,signed,typename,break,double,long,sizeof,union,case,
    dynamic_cast,mutable,static,unsigned,catch,else,namespace,static_assert,using,
    char,enum,new,static_cast,virtual,char16_t,char32_t,explict,noexcept,struct,
    void,export,nullptr,switch,volatile,class,extern,operator,template,wchar_t,
    const,false,private,this,while,constexpr,float,protected,thread_local,
    const_cast,for,public,throw,std},
    emph={map,set,multimap,multiset,unordered_map,unordered_set,
    unordered_multiset,unordered_multimap,vector,string,list,deque,
    array,stack,forwared_list,iostream,memory,shared_ptr,unique_ptr,
    random,bitset,ostream,istream,cout,cin,endl,move,default_random_engine,
    uniform_int_distribution,iterator,algorithm,functional,bing,numeric,},
    emphstyle=\color{CPPViolet}, 
}

\lstdefinestyle{Java}{
    language=[AspectJ]Java,
    keywordstyle=\color{keywordcolor}\bfseries
}

\lstdefinestyle{Python}{
    language=Python,
}


%%-------------------------------正文开始---------------------------%%
\begin{document}

%%-----------------------封面--------------------%%
\cover

%%------------------摘要-------------%%
\begin{abstract}
%
XXXXX\\
\textbf{关键词:}XXX
%
\end{abstract}

%%--------------------------目录页------------------------%%
\newpage
\tableofcontents
\thispagestyle{empty} % 目录不显示页码

%%------------------------正文页从这里开始-------------------%
\newpage
\setcounter{page}{1} % 让页码从正文开始编号

%%可选择这里也放一个标题
\begin{center}
    \title{ \Huge \textbf{} }
\end{center}


\thispagestyle{empty} % 首页不显示页码


\section{问题设定与基本概念}
给定 $n$ 个样本,特征维度为 $d$。每个样本 $X_i\in\mathbb{R}^d$,部分属于类别 $C$,其余不属于 $C$。将样本视为 $\mathbb{R}^d$ 中的点。

\textbf{判别函数/预测函数/判别式}(discriminant function)定义为标量函数 $f(x)$,满足
\[
f(x)>0\Rightarrow x\in C,\qquad f(x)\le 0\Rightarrow x\notin C.
\]
\textbf{决策边界}(decision boundary)定义为零水平集
\[
\{x\in\mathbb{R}^d: f(x)=0\}.
\]
它通常是 $\mathbb{R}^d$ 中的一个 $(d-1)$ 维曲面,即 $f$ 在等值 $0$ 处的\textbf{等值面}(isosurface);类似地还有 $\{x:f(x)=1\}$ 等其它等值面。过度贴合训练集而产生蜿蜒曲线将导致\textbf{过拟合}。

\section{线性分类器与超平面}
若取线性判别函数
\[
f(x)=w\cdot x+\alpha,
\]
则决策边界为
\[
H=\{x:w\cdot x=-\alpha\},
\]
称 $H$ 为\textbf{超平面}($d{=}2$ 为直线,$d{=}3$ 为平面)。超平面的三条核心性质:
\begin{enumerate}
  \item 维度为 $d-1$,将 $\mathbb{R}^d$ 切分为两半;
  \item 平直(由一次方程定义,无曲率);
  \item 无界(无限延伸)。
\end{enumerate}

\subsection{法向量与正交性质}
若 $x,y\in H$,则
\[
w\cdot (y-x)=0.
\]
因此 $w$ 垂直于 $H$ 上的任意方向,$w$ 称为 $H$ 的\textbf{法向量}。

\subsection{带符号距离与原点距离}
若 $w$ 为单位向量,则 $f(x)=w\cdot x+\alpha$ 是点到超平面 $H$ 的\textbf{带符号距离},正负号由法向量一侧决定;$H$ 到原点的距离为 $|\alpha|$,且 $\alpha=0$ 当且仅当 $H$ 过原点。若 $w$ 非单位向量,可将 $w,\alpha$ 同时除以 $\|w\|$ 归一化。$w$ 与 $\alpha$ 的系数统称为\textbf{权重/回归系数}(weights)。

\subsection{点到超平面的距离(通式)}
对任意 $w\ne 0$,点 $x_0$ 到 $H:\ w\cdot x+\alpha=0$ 的距离
\[
d(x_0,H)=\frac{|w\cdot x_0+\alpha|}{\|w\|}.
\]
当 $w$ 为单位向量时,$f(x)$ 即上述距离的带符号形式。

\section{质心法(Centroid Method)}
记正类与负类(非 $C$)的样本均值为
\[
\mu_C=\frac{1}{|C|}\sum_{X_i\in C}X_i,\qquad
\mu_X=\frac{1}{|X|}\sum_{X_i\notin C}X_i.
\]
采用判别函数
\[
f(x)=(\mu_C-\mu_X)\cdot x-(\mu_C-\mu_X)\cdot\frac{\mu_C+\mu_X}{2}.
\]
几何意义:法向量为 $\mu_C-\mu_X$;决策边界为连接 $\mu_C$ 与 $\mu_X$ 线段的\textbf{垂直平分超平面}。在某些数据上(即便线性可分)此方法仍可能误分;但当正负类分别来自两个高斯分布且协方差矩阵相同时,常表现良好。亦可调节标量项 $\alpha$(不改变法向量)以减少误分。

\paragraph{与贝叶斯最优分类器的联系}
若 $x|C_k\sim \mathcal{N}(\mu_k,\sigma^2 I)$ 且先验相近,则
\[
\log p(x|C_1)-\log p(x|C_2)\propto
-\tfrac{1}{2\sigma^2}\!\left(\|x-\mu_1\|^2-\|x-\mu_2\|^2\right),
\]
等价于“判给更近的质心”。此时质心法与 LDA 的线性边界一致,达到贝叶斯最优;若协方差不同,则最优边界为二次曲面(QDA)。

\section{线性可分与感知机(Perceptron)}
\textbf{线性可分}:存在某个超平面能将全部训练点正确划分。

\subsection{算法设定}
样本行向量 $X_i$ 存于矩阵 $X$ 的第 $i$ 行,标签
\[
y_i=\begin{cases}
1,& X_i\in C,\\
-1,& X_i\notin C.
\end{cases}
\]
为简便先考虑过原点的边界(后续可加偏置)。

\subsection{约束、损失与风险}
目标是找到 $w$ 使
\[
y_i(X_i\cdot w)\ge 0\quad(\forall i).
\]
定义分段线性\textbf{损失函数}
\[
L(z,y_i)=
\begin{cases}
0,& y_i z\ge 0,\\
-\,y_i z,& y_i z<0,
\end{cases}
\]
并令\textbf{风险函数}
\[
R(w)=\frac{1}{n}\sum_{i=1}^n L(X_i\cdot w,y_i)
=\frac{1}{n}\sum_{i:\,y_iX_i\cdot w<0}\!\!\!\!\big(-y_i\,X_i\cdot w\big).
\]
若 $w$ 正确分类全部样本则 $R(w)=0$;否则 $R(w)>0$,训练目标为
\[
\min_w R(w).
\]

\subsection{与梯度下降的关系与更新}
对误分类样本 $i$,有 $\nabla_w L(X_i\cdot w,y_i)=-y_i X_i$。随机梯度下降更新为
\[
w\leftarrow w-\eta\,\nabla_w L
= w+\eta\,y_i X_i,
\]
当学习率 $\eta=1$ 即经典感知机更新。感知机针对线性可分数据“慢但正确”;若数据不可分则不收敛。它不追求最大间隔,鲁棒性不如 SVM。





%%----------- 参考文献 -------------------%%
%在reference.bib文件中填写参考文献,此处自动生成
\newpage
\reference


\end{document}