%!TeX program = xelatex
\documentclass[10.5pt,hyperref,a4paper,UTF8]{ctexart}
\usepackage{RUCReport}
\usepackage{listings}
\usepackage{xcolor}
\usepackage{threeparttable}
\usepackage{booktabs}
\usepackage{array}
\usepackage{subcaption}
\usepackage{graphicx}
\usepackage{geometry}
\usepackage{float}
\geometry{a4paper, margin=1in}
\renewcommand\thesubfigure{\arabic{subfigure}} % 子图编号改为中文小写
% 定义可能使用到的颜色
\definecolor{CPPLight}  {HTML} {686868}
\definecolor{CPPSteel}  {HTML} {888888}
\definecolor{CPPDark}   {HTML} {262626}
\definecolor{CPPBlue}   {HTML} {4172A3}
\definecolor{CPPGreen}  {HTML} {487818}
\definecolor{CPPBrown}  {HTML} {A07040}
\definecolor{CPPRed}    {HTML} {AD4D3A}
\definecolor{CPPViolet} {HTML} {7040A0}
\definecolor{CPPGray}  {HTML} {B8B8B8}
\definecolor{keywordcolor}{rgb}{0.8,0.1,0.5}
\definecolor{webgreen}{rgb}{0,.5,0}
\definecolor{bgcolor}{rgb}{0.92,0.92,0.92}

\makeatletter
\renewcommand\@cite[2]{\textsuperscript{[#1]}} % 将引用变为右上角的形式
\makeatother

\hypersetup{
    citecolor=black,  % 修改文献引用的颜色为黑色
}

\lstset{
    breaklines = true,                                   % 自动将长的代码行换行排版
    extendedchars=false,                                 % 解决代码跨页时,章节标题,页眉等汉字不显示的问题
    columns=fixed,       
    numbers=left,                                        % 在左侧显示行号
    basicstyle=\zihao{-5}\ttfamily,
    numberstyle=\small,
    frame=none,                                          % 不显示背景边框
    % backgroundcolor=\color[RGB]{245,245,244},            % 设定背景颜色
    keywordstyle=\color[RGB]{40,40,255},                 % 设定关键字颜色
    numberstyle=\footnotesize\color{darkgray},           % 设定行号格式
    commentstyle=\it\color[RGB]{0,96,96},                % 设置代码注释的格式
    stringstyle=\rmfamily\slshape\color[RGB]{128,0,0},   % 设置字符串格式
    showstringspaces=false,                              % 不显示字符串中的空格
    % frame=leftline,topline,rightline, bottomline         %分别对应只在左侧,上方,右侧,下方有竖线
    frame=shadowbox,                                     % 设置阴影
    rulesepcolor=\color{red!20!green!20!blue!20},        % 阴影颜色
    basewidth=0.6em,
}

\lstdefinestyle{CPP}{
    language=c++,                                        % 设置语言
    morekeywords={alignas,continute,friend,register,true,alignof,decltype,goto,
    reinterpret_cast,try,asm,defult,if,return,typedef,auto,delete,inline,short,
    typeid,bool,do,int,signed,typename,break,double,long,sizeof,union,case,
    dynamic_cast,mutable,static,unsigned,catch,else,namespace,static_assert,using,
    char,enum,new,static_cast,virtual,char16_t,char32_t,explict,noexcept,struct,
    void,export,nullptr,switch,volatile,class,extern,operator,template,wchar_t,
    const,false,private,this,while,constexpr,float,protected,thread_local,
    const_cast,for,public,throw,std},
    emph={map,set,multimap,multiset,unordered_map,unordered_set,
    unordered_multiset,unordered_multimap,vector,string,list,deque,
    array,stack,forwared_list,iostream,memory,shared_ptr,unique_ptr,
    random,bitset,ostream,istream,cout,cin,endl,move,default_random_engine,
    uniform_int_distribution,iterator,algorithm,functional,bing,numeric,},
    emphstyle=\color{CPPViolet}, 
}

\lstdefinestyle{Java}{
    language=[AspectJ]Java,
    keywordstyle=\color{keywordcolor}\bfseries
}

\lstdefinestyle{Python}{
    language=Python,
}
\theoremstyle{definition}
\newtheorem{definition}{定义}
\newtheorem{proposition}{命题}
\newtheorem{remark}{说明}

%%-------------------------------正文开始---------------------------%%
\begin{document}

%%-----------------------封面--------------------%%
\cover

%%------------------摘要-------------%%
\begin{abstract}
%
XXXXX\\
\textbf{关键词:}XXX
%
\end{abstract}

%%--------------------------目录页------------------------%%
\newpage
\tableofcontents
\thispagestyle{empty} % 目录不显示页码

%%------------------------正文页从这里开始-------------------%
\newpage
\setcounter{page}{1} % 让页码从正文开始编号

%%可选择这里也放一个标题
\begin{center}
    \title{ \Huge \textbf{} }
\end{center}


\thispagestyle{empty} % 首页不显示页码


\section{四级抽象与相互作用}

\subsection{总体框架}
\begin{enumerate}
  \item \textbf{Application / Data(应用与数据)}。首先判断数据是否带标签。若有标签,进一步区分分类(离散标签)与回归(连续标签);若无标签,则多为聚类(相似性)或降维(定位)。
  \item \textbf{Model(模型/假设空间)}。允许哪些假设:线性/多项式/逻辑回归/神经网络/最近邻/决策树等。模型容量影响过拟合与欠拟合,也影响可解释性与推断。
  \item \textbf{Optimization Problem(优化问题)}。把任务转写成“变量+目标函数+约束”的形式。常见有:无约束最优化、凸规划、最小二乘、PCA 等。
  \item \textbf{Optimization Algorithm(优化算法)}。选择实际的求解器,如梯度下降、单纯形法、SVD 等。
\end{enumerate}

\subsection{级联效应}
\begin{enumerate}
  \item 模型不同会改变目标形式与可解性,从而改变可用的算法。
  \item 从线性分类器切换到神经网络,不仅假设空间改变,优化问题也由凸变为高度非凸,对应的算法与调参策略都要随之改变。
\end{enumerate}

\section{无约束优化与凸性}

\subsection{极小值与平滑性}
\begin{enumerate}
  \item 目标函数 $f(w)$ 连续可导且梯度连续时称为光滑(smooth)。
  \item 全局极小 $w_\star$ 满足 $f(w_\star)\le f(v)$ 对任意 $v$ 成立;局部极小在某个邻域内不可继续下降。
\end{enumerate}

\subsection{凸函数的关键性质}
\begin{enumerate}
  \item 定义:在凸域上,任意 $x,y$ 与 $\beta\in[0,1]$,有
  \[
  f\big(x+\beta(y-x)\big)\le (1-\beta)f(x)+\beta f(y).
  \]
  \item 凸函数之和仍凸;许多风险函数(如感知机风险、逻辑回归负对数似然)是多个样本损失的和,因此保持凸性。
  \item 在封闭凸域上,连续凸函数要么无下界,要么只有一个局部极小,要么是由\emph{全局等值极小点}构成的连通集合。后两种情形中,沿“下坡方向”迭代最终可达全局最优(或其等值集合)。
\end{enumerate}

\subsection{梯度下降与学习率}
\begin{enumerate}
  \item 基本迭代:$w_{t+1}=w_t-\varepsilon\nabla f(w_t)$。
  \item $\varepsilon$ 过大导致发散或震荡;过小导致收敛缓慢。一个简单的自适应策略是:若本步 $f$ 上升则减小步长。
  \item 在实践中,梯度下降通常是“趋近”极小值而不精确到达,这表现为\emph{收敛}。
\end{enumerate}

\section{多维曲率、Hessian 与病态条件}

\subsection{方向曲率与 Hessian}
\begin{definition}[Hessian]
设 $f:\mathbb{R}^d\to\mathbb{R}$ 二阶可微,Hessian 为二阶导矩阵 $H(w)=\nabla^2 f(w)$。
\end{definition}

\begin{proposition}[方向曲率的二次型表示]
对任意单位向量 $v$,令 $\phi(t)=f(w_0+t v)$,则
\[
\phi''(0)=v^\top H(w_0)\,v.
\]
\textbf{证明要点}:链式法则 $\phi'(t)=\nabla f(w_0+tv)^\top v$,再对 $t$ 求导得 $\phi''(0)=v^\top \nabla^2 f(w_0)v$;或用二阶泰勒展开取 $t^2$ 项系数即可。
\end{proposition}

\begin{proposition}[特征值等于特征方向的曲率]
当 $H$ 为对称实矩阵(由二阶混合偏导可交换得知),存在正交分解 $H=Q\Lambda Q^\top$。若取方向 $v=q_i$(第 $i$ 个特征向量),则
\[
v^\top H v=q_i^\top H q_i=\lambda_i,
\]
即对应特征值就是该方向的二阶曲率。
\end{proposition}

\begin{proposition}[“Rayleigh–Ritz”上下界的朴素推导]
任意单位 $v$ 在特征向量基下写成 $v=\sum_i \alpha_i q_i$ 且 $\sum_i \alpha_i^2=1$。于是
\[
v^\top H v=\sum_i \lambda_i \alpha_i^2.
\]
因此 $v^\top H v$ 是特征值的加权平均,必位于 $[\lambda_{\min},\lambda_{\max}]$ 区间内,端点由 $v=q_{\min}$ 或 $v=q_{\max}$ 取得。
\end{proposition}

\subsection{病态条件与学习率各向异性}
\begin{enumerate}
  \item 若 $\kappa(H)=\lambda_{\max}/\lambda_{\min}$ 很大,则不同方向曲率差异悬殊,等高线呈细长椭圆,称为病态(ill-conditioning)。
  \item 单一学习率难以同时适配所有方向:对陡峭方向过大易发散,对平缓方向过小又收敛慢。
  \item 典型缓解:特征缩放/标准化、预条件化、以及自适应学习率(Adam、RMSProp 等按坐标自调步长)。
\end{enumerate}

\subsection{鞍点}
\begin{enumerate}
  \item 鞍点是梯度为零但存在正负曲率方向并存的点,例如 $f(x,y)=x^2-y^2$ 在原点。
  \item 在非凸问题(神经网络)中常见;SGD 的噪声有助于逃离部分鞍点。
\end{enumerate}

\section{线性规划(LP):线性目标 + 线性不等式}

\subsection{形式与几何}
\begin{enumerate}
  \item 形式:$\max/\min\; c^\top w\ \text{s.t.}\ Aw\le b$。可行域是凸多面体。
  \item \emph{活跃约束}:在最优解处以等号成立的约束。可能存在多重最优(如目标方向与某边界平行时一整条边都最优);所有最优解的集合仍是凸的。
\end{enumerate}

\subsection{与“线性可分”的对应}
\begin{enumerate}
  \item 存在线性分类器等价于可行域非空:寻找 $(w,\alpha)$ 使 $y_i\,(w^\top x_i+\alpha)\ge 1$ 对所有 $i$。
  \item LP 的挑战是确定最终会成为活跃的约束组合,组合数量呈指数级;常用\textbf{单纯形法}在多面体顶点间沿边移动直至最优。
\end{enumerate}

\section{二次规划(QP)与最大间隔 SVM}

\subsection{QP 基本形态与可解性}
\begin{enumerate}
  \item 形式:$\min\; f(w)=w^\top Q w + c^\top w\ \text{s.t.}\ Aw\le b$,其中 $Q$ 对称半正定则凸可解,$Q$ 正定则解唯一。
  \item 若 $Q$ 不定,目标非凸,普遍为 NP-hard。
\end{enumerate}

\subsection{从“可分”到“最大间隔”的推导}
\begin{enumerate}
  \item 目标是最大化几何间隔
  \[
  \gamma=\min_i \frac{y_i\,(w^\top x_i+b)}{\lVert w\rVert}.
  \]
  \item 通过缩放标准化约束 $y_i(w^\top x_i+b)\ge 1$,此时 $\gamma=1/\lVert w\rVert$。
  \item 因而“最大间隔”等价于
  \[
  \min_{w,b}\ \tfrac12\lVert w\rVert^2\quad \text{s.t.}\quad y_i(w^\top x_i+b)\ge 1,
  \]
  即二次目标 + 线性不等式约束的 QP。
  \item 最优点处贴边的样本构成\emph{支持向量},是几何上的活跃约束。
\end{enumerate}

\subsection{软间隔与合页损失的等价}
\begin{enumerate}
  \item 带松弛变量的带约束形式:
  \[
  \min_{w,b,\xi}\ \tfrac12\lVert w\rVert^2 + C\sum_{i=1}^n \xi_i\quad
  \text{s.t.}\quad y_i(w^\top x_i+b)\ge 1-\xi_i,\ \xi_i\ge 0.
  \]
  \item 由约束得 $\xi_i\ge \max(0,\,1-y_i(w^\top x_i+b))$;由于目标对 $\xi_i$ 单调,在最优处取下界,故等价为\emph{无约束}:
  \[
  \min_{w,b}\ \tfrac12\lVert w\rVert^2+C\sum_{i=1}^n \max\bigl(0,\,1-y_i(w^\top x_i+b)\bigr),
  \]
  第二项即合页损失(hinge loss)。
\end{enumerate}

\subsection{核技巧一言以蔽之}
\begin{enumerate}
  \item 将内积 $x^\top z$ 替换为核函数 $K(x,z)=\langle \Phi(x),\Phi(z)\rangle$,在隐式高维中作线性分类,从而得到非线性判别边界。
\end{enumerate}

\subsection{QP 求解算法与对比}
\begin{enumerate}
  \item \textbf{单纯形类(Simplex-like)}:适用广、能精确解一般 QP;但大规模慢、内存压力大;更偏“组合/离散”风格。
  \item \textbf{SMO(Sequential Minimal Optimization)}:把对偶 QP 分解成反复求解 $2$ 变量子问题;中等规模核 SVM 高效,LIBSVM 与 \texttt{sklearn.SVC} 采用;但超大规模或核矩阵存储时仍受限。
  \item \textbf{坐标下降(Coordinate Descent)}:逐坐标优化,极擅长超大规模稀疏线性 SVM(文本/点击率等),LIBLINEAR 与 \texttt{LinearSVC} 采用;但不直接适配核 SVM,收敛速度依赖坐标选择策略。
\end{enumerate}

\section{实践要点与易错清单}

\subsection{学习率与收敛}
\begin{enumerate}
  \item 一维:步长过大发散,过小缓慢;多维病态时单一步长无法兼顾各方向。
  \item 简便诊断:若某步 $f$ 上升,缩小步长;使用学习率衰减或自适应策略(Adam、RMSProp)。
\end{enumerate}

\subsection{数据与特征处理}
\begin{enumerate}
  \item 线性模型前建议标准化特征;对病态问题可考虑预条件化。
  \item 核 SVM 的核宽度、正则系数 $C$ 需网格或贝叶斯调参;大数据优先线性 SVM 或近似核方法。
\end{enumerate}

\subsection{从 LP 到 QP 的识别技巧}
\begin{enumerate}
  \item 只需“可分”的分类 $\Rightarrow$ LP 可行性问题:判断不等式系统是否有解。
  \item 追求“最大间隔” $\Rightarrow$ QP:最小化 $\frac12\lVert w\rVert^2$ 加线性约束。
  \item 不可分且希望鲁棒 $\Rightarrow$ 软间隔 + 合页损失(等价无约束)。
\end{enumerate}

\section{概念速查表}

\begin{center}
\renewcommand{\arraystretch}{1.3}
\begin{tabular}{p{3cm}p{11cm}}
\toprule
术语 & 要点 \\
\midrule
局部/全局极小 & 全局对所有点最小;局部在邻域内最小 \\
凸函数 & 连线在图像之上;凸和仍凸;许多经验风险是凸的 \\
Hessian & 二阶导矩阵;方向曲率为 $v^\top H v$ \\
病态(条件数) & $\kappa=\lambda_{\max}/\lambda_{\min}\gg 1$,不同方向曲率差异大 \\
鞍点 & 梯度为零但存在正负曲率方向并存 \\
LP & 线性目标+线性不等式;可行域多面体;单纯形法 \\
QP & 二次凸目标+线性不等式;$Q\succeq 0$ 则凸可解 \\
SVM(硬间隔) & $\min \frac12\lVert w\rVert^2\ \text{s.t.}\ y_i(w^\top x_i+b)\ge 1$ \\
SVM(软间隔) & 加松弛 $\xi_i$;等价于 $\frac12\lVert w\rVert^2+C\sum \max(0,1-y_if_i)$ \\
核技巧 & 用 $K(x,z)=\langle \Phi(x),\Phi(z)\rangle$ 替代内积 \\
SMO & 核 SVM 常用;逐个两变量子问题精解 \\
坐标下降 & 线性 SVM 大规模稀疏数据利器 \\
\bottomrule
\end{tabular}
\end{center}

\section{关键推导与证明摘要}

\subsection*{A. 方向曲率 = 二次型}
设 $\phi(t)=f(w_0+t v)$,链式法则给出 $\phi''(0)=v^\top \nabla^2 f(w_0)\,v$。因此 Hessian 的特征值是特征方向上的曲率;任意方向曲率为特征值的非负加权平均,从而被夹在 $[\lambda_{\min},\lambda_{\max}]$ 内。

\subsection*{B. 软间隔 SVM $\Rightarrow$ 合页损失}
约束 $y_i(w^\top x_i+b)\ge 1-\xi_i$ 与 $\xi_i\ge 0$ 蕴含 $\xi_i\ge \max(0,1-y_if_i)$。目标对 $\xi_i$ 单调,最优取下界,代回得 $\min\ \tfrac12\lVert w\rVert^2+C\sum \max(0,1-y_if_i)$。





%%----------- 参考文献 -------------------%%
%在reference.bib文件中填写参考文献,此处自动生成
\newpage
\reference


\end{document}