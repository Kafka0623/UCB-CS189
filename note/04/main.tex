%!TeX program = xelatex
\documentclass[10.5pt,hyperref,a4paper,UTF8]{ctexart}
\usepackage{RUCReport}
\usepackage{listings}
\usepackage{xcolor}
\usepackage{threeparttable}
\usepackage{booktabs}
\usepackage{array}
\usepackage{subcaption}
\usepackage{graphicx}
\usepackage{geometry}
\usepackage{float}
\geometry{a4paper, margin=1in}
\renewcommand\thesubfigure{\arabic{subfigure}} % 子图编号改为中文小写
% 定义可能使用到的颜色
\definecolor{CPPLight}  {HTML} {686868}
\definecolor{CPPSteel}  {HTML} {888888}
\definecolor{CPPDark}   {HTML} {262626}
\definecolor{CPPBlue}   {HTML} {4172A3}
\definecolor{CPPGreen}  {HTML} {487818}
\definecolor{CPPBrown}  {HTML} {A07040}
\definecolor{CPPRed}    {HTML} {AD4D3A}
\definecolor{CPPViolet} {HTML} {7040A0}
\definecolor{CPPGray}  {HTML} {B8B8B8}
\definecolor{keywordcolor}{rgb}{0.8,0.1,0.5}
\definecolor{webgreen}{rgb}{0,.5,0}
\definecolor{bgcolor}{rgb}{0.92,0.92,0.92}

\makeatletter
\renewcommand\@cite[2]{\textsuperscript{[#1]}} % 将引用变为右上角的形式
\makeatother

\hypersetup{
    citecolor=black,  % 修改文献引用的颜色为黑色
}

\lstset{
    breaklines = true,                                   % 自动将长的代码行换行排版
    extendedchars=false,                                 % 解决代码跨页时,章节标题,页眉等汉字不显示的问题
    columns=fixed,       
    numbers=left,                                        % 在左侧显示行号
    basicstyle=\zihao{-5}\ttfamily,
    numberstyle=\small,
    frame=none,                                          % 不显示背景边框
    % backgroundcolor=\color[RGB]{245,245,244},            % 设定背景颜色
    keywordstyle=\color[RGB]{40,40,255},                 % 设定关键字颜色
    numberstyle=\footnotesize\color{darkgray},           % 设定行号格式
    commentstyle=\it\color[RGB]{0,96,96},                % 设置代码注释的格式
    stringstyle=\rmfamily\slshape\color[RGB]{128,0,0},   % 设置字符串格式
    showstringspaces=false,                              % 不显示字符串中的空格
    % frame=leftline,topline,rightline, bottomline         %分别对应只在左侧,上方,右侧,下方有竖线
    frame=shadowbox,                                     % 设置阴影
    rulesepcolor=\color{red!20!green!20!blue!20},        % 阴影颜色
    basewidth=0.6em,
}

\lstdefinestyle{CPP}{
    language=c++,                                        % 设置语言
    morekeywords={alignas,continute,friend,register,true,alignof,decltype,goto,
    reinterpret_cast,try,asm,defult,if,return,typedef,auto,delete,inline,short,
    typeid,bool,do,int,signed,typename,break,double,long,sizeof,union,case,
    dynamic_cast,mutable,static,unsigned,catch,else,namespace,static_assert,using,
    char,enum,new,static_cast,virtual,char16_t,char32_t,explict,noexcept,struct,
    void,export,nullptr,switch,volatile,class,extern,operator,template,wchar_t,
    const,false,private,this,while,constexpr,float,protected,thread_local,
    const_cast,for,public,throw,std},
    emph={map,set,multimap,multiset,unordered_map,unordered_set,
    unordered_multiset,unordered_multimap,vector,string,list,deque,
    array,stack,forwared_list,iostream,memory,shared_ptr,unique_ptr,
    random,bitset,ostream,istream,cout,cin,endl,move,default_random_engine,
    uniform_int_distribution,iterator,algorithm,functional,bing,numeric,},
    emphstyle=\color{CPPViolet}, 
}

\lstdefinestyle{Java}{
    language=[AspectJ]Java,
    keywordstyle=\color{keywordcolor}\bfseries
}

\lstdefinestyle{Python}{
    language=Python,
}

\newcommand{\R}{\mathbb{R}}
\newcommand{\norm}[1]{\left\lVert #1\right\rVert}
\newcommand{\inner}[2]{\left\langle #1,#2\right\rangle}

\theoremstyle{definition}
\newtheorem{definition}{定义}
\newtheorem{theorem}{定理}
\newtheorem{proposition}{命题}
\newtheorem{remark}{说明}
\newtheorem{example}{例}
%%-------------------------------正文开始---------------------------%%
\begin{document}

%%-----------------------封面--------------------%%
\cover

%%------------------摘要-------------%%
\begin{abstract}
%
XXXXX\\
\textbf{关键词:}XXX
%
\end{abstract}

%%--------------------------目录页------------------------%%
\newpage
\tableofcontents
\thispagestyle{empty} % 目录不显示页码

%%------------------------正文页从这里开始-------------------%
\newpage
\setcounter{page}{1} % 让页码从正文开始编号

%%可选择这里也放一个标题
\begin{center}
    \title{ \Huge \textbf{} }
\end{center}


\thispagestyle{empty} % 首页不显示页码


\section{问题背景与动机}
\subsection{硬间隔 SVM 的局限}
硬间隔 SVM 追求在训练集上严格线性可分并最大化间隔,但
\begin{enumerate}
  \item 当数据不可线性可分时无可行解;
  \item 对离群点高度敏感:单一点可显著改变最大间隔超平面;
  \item 过于强调零训练误差,泛化受损。
\end{enumerate}

\subsection{软间隔思想}
允许少量样本违反间隔(甚至被误分),用\emph{松弛变量}刻画违约程度,在目标函数中惩罚之。几何上仍保持较宽的间隔,从而提升鲁棒性。

\section{软间隔 SVM:约束、几何与优化}
\subsection{松弛变量与几何解释}
给定样本 $\{(x_i,y_i)\}_{i=1}^n$, $x_i\in\R^d$, $y_i\in\{\pm1\}$,判别函数
\[
f(x)=\bm w^\top x+\alpha.
\]
引入 $\xi_i\ge0$,把硬间隔约束
\[
y_i(\bm w^\top x_i+\alpha)\ge1
\]
放宽为
\[
y_i(\bm w^\top x_i+\alpha)\ge1-\xi_i,\qquad \xi_i\ge0.
\]
几何分类:
\begin{enumerate}
  \item $\xi_i=0$:样本在正确一侧且在间隔之外;
  \item $0<\xi_i<1$:样本落入间隔区但仍正确分类;
  \item $\xi_i\ge1$:样本被误分。
\end{enumerate}
软间隔的宽度仍定义为 $2/\norm{\bm w}$;与硬间隔不同,最近训练点可侵入间隔。

\subsection{原始问题(QP)与参数 \(C\)}
\begin{equation}\label{eq:primal}
\begin{aligned}
\min_{\bm w,\alpha,\bm\xi}\quad & \frac{1}{2}\norm{\bm w}^2 + C\sum_{i=1}^n \xi_i\\
\text{s.t.}\quad & y_i(\bm w^\top x_i+\alpha)\ge 1-\xi_i,\ i=1,\dots,n,\\
& \xi_i\ge0,\ i=1,\dots,n.
\end{aligned}
\end{equation}
式 \eqref{eq:primal} 为二次规划:目标二次、约束线性。\emph{正则化超参数} $C>0$ 在“最大化间隔”与“最小化训练误差”之间做权衡。
\begin{center}
\begin{tabular}{@{}lll@{}}
\toprule
& $C$ 小 & $C$ 大\\
\midrule
期望 & 更大间隔 $1/\norm{\bm w}$ & 使大多数 $\xi_i$ 逼近 $0$\\
风险 & 欠拟合(训练误差偏大) & 过拟合(对噪声敏感)\\
对离群点 & 不敏感 & 极敏感\\
\bottomrule
\end{tabular}
\end{center}
选择 $C$ 通常用验证(交叉验证)完成。

\subsection{偏置与同质化}
原空间中 $f(x)=\bm w^\top x+\alpha$;同质化升维为
\[
x'=(x,1),\quad w'=(\bm w,\alpha),\quad f(x)=w'^\top x'.
\]
注意 $\,\bm w^\top(x+\alpha)\,$ 无意义(维度不匹配)。

\section{非线性边界:特征与升维}
\subsection{总路线}
问题:在原空间不可线性可分。策略:构造非线性特征映射 $\Phi:\R^d\to\R^D$,在高维 $\Phi$-空间用\emph{线性}分类器,得到原空间的\emph{非线性}边界。

\subsection{抛物提升映射:球面 \(\leftrightarrow\) 超平面}
取
\[
\Phi(x)=\begin{bmatrix}x\\ \norm{x}^2\end{bmatrix}\in\R^{d+1}.
\]
\begin{theorem}[球 \(\leftrightarrow\) 超平面等价]
设原空间球 $\{x:\norm{x-c}^2<\rho^2\}$。则
\[
\norm{x-c}^2<\rho^2
\iff
\underbrace{\begin{bmatrix}-2c^\top&1\end{bmatrix}}_{\text{法向量}}
\underbrace{\begin{bmatrix}x\\ \norm{x}^2\end{bmatrix}}_{\Phi(x)}
<\rho^2-\norm{c}^2,
\]
即“球内”$\Leftrightarrow$“升维后超平面之下”。超平面是半径 $\to\infty$ 的球的退化情形。
\end{theorem}

\subsection{二次曲面(Quadrics)与二次特征}
三维的一般二次曲面
\[
Ax_1^2+Bx_2^2+Cx_3^2+Dx_1x_2+Ex_2x_3+Fx_3x_1+Gx_1+Hx_2+Ix_3+\alpha=0.
\]
令
\[
\Phi(x)=[x_1^2,x_2^2,x_3^2,\ x_1x_2,x_2x_3,x_3x_1,\ x_1,x_2,x_3]^\top,
\]
则任意二次边界可写为线性形式 $f(x)=\bm w^\top\Phi(x)+\alpha$。当 $d$ 大时交叉项数量为 $\mathcal{O}(d^2)$,计算代价显著;若仅保留平方项 $x_j^2$,新增维度线性增长,但边界\emph{只能与坐标轴对齐}(不可任意旋转)。

\subsection{一般 $p$ 次多项式边界与升维的普适性}
任意次数 $\le p$ 的多项式
\[
f(x)=\sum_{|\alpha|\le p} c_\alpha\,x^\alpha,\qquad x^\alpha=\prod_{j=1}^dx_j^{\alpha_j},
\]
通过构造 $\Phi(x)$ 为所有次数 $\le p$ 的单项式集合(维度 $\binom{d+p}{p}$),都有
\[
f(x)=\bm w^\top\Phi(x),
\]
从而在 $\Phi$-空间线性化。这一结论说明:\emph{所有多项式形状的边界皆可通过合适升维变为超平面}。但维度随 $p$ 呈 $\mathcal{O}(d^p)$ 爆炸。

\subsection{核技巧(Kernel Trick)}
对偶形式仅依赖 $\inner{\Phi(x_i)}{\Phi(x)}$。定义核函数
\[
K(x,z)=\inner{\Phi(x)}{\Phi(z)},
\]
即可无需显式计算 $\Phi$ 而在高维求解。典型示例:多项式核 $K(x,z)=(x^\top z+c)^p$、RBF 核等。核化使高阶多项式在计算上可行。

\section{次数、间隔与泛化:复杂度权衡}
\subsection{提高次数的双重效应}
\begin{enumerate}
  \item 充分高的次数可使原本不可分的数据在高维变为\emph{线性可分};
  \item 次数升高常见到\emph{更宽的有效间隔}(在合适的正则下),但过高会过拟合。
\end{enumerate}

\subsection{训练与测试误差随复杂度的典型曲线}
随模型复杂度(例如次数 $p$ 或等效自由度)上升,训练误差单调下降;测试误差先降后升。最佳点位于两者平衡处(例如二次决策在示例中给出最小测试误差)。

\subsection{与 \(C\) 的联动调参}
若同时使用多项式特征与软间隔 SVM,需要联合选择
\begin{enumerate}
  \item 多项式次数 $p$(控制表示能力);
  \item 正则化超参数 $C$(控制容错与间隔)。
\end{enumerate}
一般地,不同 $p$ 下的最优 $C$ 不同,改变 $p$ 时应重新验证 $C$。

\section{特征工程:从多项式到任务特定特征}
\subsection{边缘检测与 HOG}
在图像任务中,直接使用原始像素对光照/位置/噪声敏感。可先用 Sobel/方向性高斯导数等算子近似梯度,再在局部网格统计\emph{梯度方向直方图}(如每块 12 个方向 bins),拼接为向量作为特征(HOG),再交由线性 SVM 训练。实证对比:像素+线性 SVM 的测试错误率远高于 HOG+线性 SVM(错误率显著下降)。
\begin{remark}
该案例表明:合适的任务特征往往比单纯提升模型复杂度更能改善泛化。
\end{remark}

\section{实践要点与建议}
\begin{enumerate}
  \item 先做标准化与特征工程,必要时再用核方法提升表示力;
  \item 用交叉验证\emph{联合}调参:次数/核参数与 \(C\);
  \item 关注鲁棒性:数据含噪/离群点时优先软间隔;
  \item 维度-计算量-过拟合三者权衡:去交叉项可显著降维,代价是边界受限(轴对齐)。
\end{enumerate}

\section*{附录A:对偶与核化}

\subsection*{A.1 原始问题与拉格朗日函数}
软间隔 SVM 的原始(Primal)二次规划为
\begin{equation}\label{eq:primal_full}
\begin{aligned}
\min_{\bm w,\ \alpha,\ \bm\xi}\quad & \frac{1}{2}\|\bm w\|^2 \ +\ C\sum_{i=1}^n \xi_i\\
\text{s.t.}\quad & y_i(\bm w^\top x_i + \alpha)\ \ge\ 1-\xi_i,\quad i=1,\dots,n,\\
& \xi_i \ \ge\ 0,\quad i=1,\dots,n,
\end{aligned}
\end{equation}
其中 $x_i\in\mathbb{R}^d,\ y_i\in\{\pm1\}$,$C>0$ 为正则化超参数。引入拉格朗日乘子
\[
\lambda_i\ \ge\ 0\quad \text{对应约束 } y_i(\bm w^\top x_i+\alpha)\ge 1-\xi_i,\qquad
\mu_i\ \ge\ 0\quad \text{对应约束 } \xi_i\ge 0.
\]
拉格朗日函数
\begin{align}
\mathcal{L}(\bm w,\alpha,\bm\xi;\bm\lambda,\bm\mu)
&=\frac{1}{2}\|\bm w\|^2 + C\sum_{i=1}^n \xi_i
- \sum_{i=1}^n \lambda_i\bigl[y_i(\bm w^\top x_i+\alpha)-1+\xi_i\bigr]
- \sum_{i=1}^n \mu_i \xi_i.\label{eq:Lagrangian}
\end{align}

\subsection*{A.2 KKT 条件与驻点}
对 $\bm w,\alpha,\xi_i$ 的一阶驻点条件(Stationarity):
\begin{align}
\frac{\partial \mathcal{L}}{\partial \bm w} &= \bm w - \sum_{i=1}^n \lambda_i y_i x_i = \bm 0
\ \ \Rightarrow\ \ \bm w^\star = \sum_{i=1}^n \lambda_i y_i x_i, \label{eq:w_star}\\
\frac{\partial \mathcal{L}}{\partial \alpha} &= -\sum_{i=1}^n \lambda_i y_i = 0
\ \ \Rightarrow\ \ \sum_{i=1}^n \lambda_i y_i = 0,\label{eq:alpha_cond}\\
\frac{\partial \mathcal{L}}{\partial \xi_i} &= C - \lambda_i - \mu_i = 0
\ \ \Rightarrow\ \ 0 \le \lambda_i \le C,\ \ \mu_i = C-\lambda_i.\label{eq:xi_cond}
\end{align}
再加上原始可行性(Primal feasibility)
\[
y_i(\bm w^\top x_i+\alpha)\ge 1-\xi_i,\quad \xi_i\ge 0,
\]
对偶可行性(Dual feasibility)
\(
\lambda_i\ge0,\ \mu_i\ge0\ (\Rightarrow 0\le\lambda_i\le C),
\)
以及互补松弛(Complementary slackness):
\begin{align}
\lambda_i\bigl[y_i(\bm w^\top x_i+\alpha)-1+\xi_i\bigr] &= 0,\label{eq:cs1}\\
\mu_i \xi_i &= 0. \label{eq:cs2}
\end{align}
由于问题满足 Slater 条件(取 $\bm w=\bm 0,\ \alpha=0,\ \xi_i=2$ 即严格可行),强对偶成立,KKT 为充要最优条件。

\subsection*{A.3 消元得到对偶问题}
将 \eqref{eq:w_star} 代入 \eqref{eq:Lagrangian} 并最小化掉原始变量,得到\emph{对偶(Dual)}:
\begin{equation}\label{eq:dual}
\begin{aligned}
\max_{\bm\lambda}\quad & 
\sum_{i=1}^n \lambda_i \ -\ \frac{1}{2}\sum_{i=1}^n\sum_{j=1}^n 
\lambda_i \lambda_j\, y_i y_j\, (x_i^\top x_j)\\
\text{s.t.}\quad &
\sum_{i=1}^n \lambda_i y_i = 0,\qquad 0\le \lambda_i \le C,\ \ i=1,\dots,n.
\end{aligned}
\end{equation}
该目标函数关于 $\bm\lambda$ 为严格凹函数(负半定二次型),带线性约束的凹优化,标准 QP 可用 SMO 等算法求解。

\subsection*{A.4 偏置项的恢复与判别函数}
求得最优 $\bm\lambda^\star$ 后,由 \eqref{eq:w_star} 有
\[
\bm w^\star = \sum_{i=1}^n \lambda_i^\star y_i x_i.
\]
对任一\emph{边界支持向量} $k$(满足 $0<\lambda_k^\star<C$,此时 \eqref{eq:cs1} 给出 $y_k(\bm w^{\star\top}x_k+\alpha^\star)=1$),可恢复偏置
\begin{equation}\label{eq:b_recover}
\alpha^\star = y_k - \sum_{i=1}^n \lambda_i^\star y_i\, (x_i^\top x_k).
\end{equation}
实践中可对所有 $0<\lambda_k^\star<C$ 的样本求得的 \eqref{eq:b_recover} 取平均以减小数值误差。
最终判别函数为
\begin{equation}\label{eq:decision}
f(x) = \operatorname{sign}\!\left(\sum_{i=1}^n \lambda_i^\star y_i\, (x_i^\top x) + \alpha^\star\right).
\end{equation}
分类几何解释:间隔宽度为 $2/\|\bm w^\star\|$。样本类型与 KKT 的对应关系:
\begin{enumerate}
\item $\lambda_i^\star=0 \Rightarrow \xi_i^\star=0,\ y_i(\bm w^{\star\top}x_i+\alpha^\star)\ge 1$:在间隔外,\emph{非}支持向量;
\item $0<\lambda_i^\star<C \Rightarrow \xi_i^\star=0,\ y_i(\bm w^{\star\top}x_i+\alpha^\star)=1$:\emph{边界}支持向量;
\item $\lambda_i^\star=C \Rightarrow y_i(\bm w^{\star\top}x_i+\alpha^\star)=1-\xi_i^\star\le 1$:在间隔内或被误分的\emph{错误/软}支持向量。
\end{enumerate}

\subsection*{A.5 与合页损失(Hinge Loss)的等价}
原始问题 \eqref{eq:primal_full} 可改写为正则化风险最小化:
\begin{equation}\label{eq:hinge_form}
\min_{\bm w,\alpha}\ \frac{1}{2}\|\bm w\|^2 + C\sum_{i=1}^n \ell_{\text{hinge}}\!\left(y_i,\ \bm w^\top x_i+\alpha\right),
\quad
\ell_{\text{hinge}}(y,t)=\max\{0,\ 1-y\,t\}.
\end{equation}
证明要点:对给定 $(\bm w,\alpha)$,对每个 $i$ 的最优 $\xi_i$ 来自
\[
\min_{\xi_i\ge0}\ \ C\xi_i\quad \text{s.t.}\ \ \xi_i \ge 1-y_i(\bm w^\top x_i+\alpha),
\]
其最优解为 $\xi_i^\star=\max\{0,\ 1-y_i(\bm w^\top x_i+\alpha)\}$,代回得到 \eqref{eq:hinge_form}。
这说明 SVM 是“二范数正则化 + 合页损失”的最大间隔分类器。

\subsection*{A.6 核化(Kernelization)}
在对偶 \eqref{eq:dual} 与判别 \eqref{eq:decision} 中,数据仅以内积 $x_i^\top x_j$、$x_i^\top x$ 出现。定义\emph{核函数}
\[
K(x,z) = \inner{\Phi(x)}{\Phi(z)},
\]
其中 $\Phi(\cdot)$ 是(可能高维/无限维的)特征映射,且 $K$ 对称且正定。将所有内积替换为核函数:
\begin{equation}\label{eq:dual_kernel}
\begin{aligned}
\max_{\bm\lambda}\quad & 
\sum_{i=1}^n \lambda_i \ -\ \frac{1}{2}\sum_{i=1}^n\sum_{j=1}^n 
\lambda_i \lambda_j\, y_i y_j\, K(x_i,x_j)\\
\text{s.t.}\quad &
\sum_{i=1}^n \lambda_i y_i = 0,\qquad 0\le \lambda_i \le C.
\end{aligned}
\end{equation}
\begin{equation}\label{eq:decision_kernel}
f(x) = \operatorname{sign}\!\left(\sum_{i=1}^n \lambda_i^\star y_i\, K(x_i,x) + \alpha^\star\right).
\end{equation}
常见核:线性核 $K(x,z)=x^\top z$、多项式核 $K(x,z)=(x^\top z+c)^p$、RBF 核 $K(x,z)=\exp\!\bigl(-\|x-z\|^2/(2\sigma^2)\bigr)$ 等。核化避免了显式构造 $\Phi$ 及其维度爆炸。

\subsection*{A.7 硬间隔与极限情形}
当数据严格可分且取 $C\to\infty$ 时,软间隔对偶 \eqref{eq:dual_kernel} 退化到硬间隔的对偶,最优解强制 $\xi_i^\star=0$,得到经典最大间隔分离超平面。

\subsection*{A.8 计算与实现要点}
1)对偶问题是带盒约束与一个等式约束的凹 QP,SMO 类分解方法高效稳定;2)偏置 $\alpha^\star$ 建议在所有 $0<\lambda_i^\star<C$ 的边界支持向量上取平均;3)特征缩放与核参数(如多项式次数、RBF 的 $\sigma$)必须与 $C$ 一并经验证优化。

\bigskip
\noindent\textbf{小结:}
软间隔 SVM 的对偶仅依赖样本的成对相似度(内积/核),最优解由少量支持向量决定;核技巧在不显式升维的情形下实现了高维线性分割,从而在原空间产生非线性决策边界。


\section*{附录B:常见核与特征维度增长}
\begin{enumerate}
  \item 多项式核 $K(x,z)=(x^\top z+c)^p$,对应显式特征维度 $\binom{d+p}{p}$;
  \item RBF 核 $K(x,z)=\exp(-\tfrac{\norm{x-z}^2}{2\sigma^2})$,对应无限维特征;
  \item 仅平方项时新增维度 $\mathcal{O}(d)$,包含交叉项时为 $\mathcal{O}(d^2)$。
\end{enumerate}




%%----------- 参考文献 -------------------%%
%在reference.bib文件中填写参考文献,此处自动生成
\newpage
\reference


\end{document}